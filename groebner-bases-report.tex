% Created 2021-04-14 Wed 10:03
% Intended LaTeX compiler: pdflatex
\documentclass[a4paper,11pt,twoside]{article}
\usepackage[utf8]{inputenc}
\usepackage[T1]{fontenc}
\usepackage{graphicx}
\usepackage{grffile}
\usepackage{longtable}
\usepackage{wrapfig}
\usepackage{rotating}
\usepackage[normalem]{ulem}
\usepackage{amsmath}
\usepackage{textcomp}
\usepackage{amssymb}
\usepackage{capt-of}
\usepackage{hyperref}
\usepackage{tikz}
\usepackage{minted}
\IfFileExists{/home/ryan/Templates/Org_Mode_Report/resources/style.sty}{\usepackage{$HOME/Templates/Org_Mode_Report/resources/style}}{}
\IfFileExists{$HOME/Templates/Org_Mode_Report/resources/referencing.sty}{\usepackage{$HOME/Templates/Org_Mode_Report/resources/referencing}}{}
\addbibresource{./ref.bib}
\usepackage[mode=buildnew]{standalone}
\usepackage{tikz}
\usetikzlibrary{decorations.fractals}
\usetikzlibrary{lindenmayersystems}
\author{Ryan Greenup}
\date{\today}
\title{Groebner Bases}
\hypersetup{
 pdfauthor={Ryan Greenup},
 pdftitle={Groebner Bases},
 pdfkeywords={},
 pdfsubject={},
 pdfcreator={Emacs 28.0.50 (Org mode 9.4.4)}, 
 pdflang={English}}
\begin{document}

\maketitle

Groebner bases is one of the main practical tools for solving systems
of polynomial equations.

\section{Summary}
\label{sec:org73c5bea}
Much of the theory of Groebner Basis is buried under needless
amounts of abstract algebra, this is for the most part unnecessary
and if I were to begin this investigation again I would first
implement Buchberger's Algorithm, manually, using \emph{Sympy} by referring to:

\begin{itemize}
\item Lectures 14 and 15 of Andreas Schulz OCW Integer Programming
Course \cite{andreasschulzIntegerProgrammingCombinatorial}
\item Chapters 1-2 of \emph{Ideals, Varieties and Algorithms} \cite{coxIdealsVarietiesAlgorithms1997}
\item Lecture 15 of Judy Holdenner's course on Algebraic Geometry \cite{judyholdenerAlgebraicGeometry2013}
\item Lecture 14 of Pablo Parrilo's course on Algebraic Techniques \cite{pabloparriloAlgebraicTechniquesSemidefinite}
\item The \emph{Sympy} source code for:
\begin{itemize}
\item \texttt{polys.roebnertools} \cite{sympydevelopmentteamSympyPolysGroebnertools}
\item \texttt{solvers.polysys} \cite{sympydevelopmentteamSympySolversPolysys}
\end{itemize}
\item The \emph{Sympy} documentation for \emph{Polynomial Manipulation} \cite{sympydevelopmentteamGrobnerBasesTheir2021}
\end{itemize}

Unfortunately this was not an option for me as these resources were
not known to me until very late in the investigation. I hope that
this report can serve as a guide for others who pick up this topic
such that they can:

\begin{itemize}
\item Come to grips with the core concepts and practical applications
quickly without wasting time on abstract algebra that is poorly
explained \footnote{In the absence of better materials a lot of time was wasted
(yes, wasted, not spent) on complex algebraic concepts when all I
needed was an algorithm to experiment with, an algorithm that the
complex texts would not provide.}
\item Identify useful resources that are written well and written with
accessibility in mind
\item Avoid material that serves as, for lack of a better word, as a
red herring.
\end{itemize}

Although \texttt{sympy} is probably not the best tool for studying
commutative algebra specifically (and the implementation may not be
battle tested either, see e.g. \cite{WrongGroebnerBasis}), the simple
and accessible nature of \texttt{sympy} made it's documentation by far the
most valuable resource for grappling with this topic.

An extension to this investigation would be to:

\begin{itemize}
\item Try and implement Buchberger's Algorithm from scratch using
functions and iterations in \emph{Python} in order to return a Reduced
Groebner Basis
\begin{itemize}
\item See Definition 4 of \cite[\S 7]{coxIdealsVarietiesAlgorithms1997}
\end{itemize}
\item Try to demonstrate, in good detail, the relationship between
the Euclidean Algorithm and Buchberger's Algorithm
\begin{itemize}
\item See \cite[p. 95]{coxIdealsVarietiesAlgorithms1997}
\end{itemize}
\item Try to implement Buchberger's Algorithm using \emph{Normal Selection
Strategy} \cite[\S 3.1.2]{hibiGrobnerBasesStatistics2014}, see also \cite{sympydevelopmentteamSympyPolysGroebnertools,prof.berndsturmfelsIntroductionGrobnerBases2017}.
\end{itemize}

\subsection{Further Resources}
\label{sec:org91084bd}
The following resources may be useful as reference material, but I
would advice against using these as any sort of primary material,
in order of recommendation (but not necessarily relevance)



\begin{enumerate}
\item Judson, T. W., \& Open Textbook Library, Abstract algebra
theory and applications \cite{judsonAbstractAlgebraTheory2016}
\item Howlett, R., An undergraduate course in Abstract Algebra:
Course notes for MATH3002 \cite{roberthowlettUndergraduateCourseAbstract}
\item Lee, G., Abstract Algebra \cite{gregoryleeAbstractAlgebra2018}
\item Grillet, P. A., Abstract Algebra \cite{grilletAbstractAlgebra2007}
\item Hibi, T., Grobner Bases: Statistics and Software Systems \cite{hibiGrobnerBasesStatistics2014}
\item Adams, W. W., \& Loustaunau, P., An introduction to Gröbner bases \cite{adamsIntroductionGrobnerBases1994}
\item Nicodemi, O., Sutherland, M. A., \& Towsley, G. W., An
introduction to abstract algebra with notes to the future
teacher \cite{nicodemiIntroductionAbstractAlgebra2007a}
\end{enumerate}


Further material that I haven't had a chance to look throughA:
includes:

\begin{itemize}
\item Becker, T., Weispfenning, V., \& Kredel, H., Gröbner bases: a computational approach to commutative algebra \cite{beckerGrobnerBasesComputational1993}
\end{itemize}

\section{Introduction}
\label{sec:org381cde0}
A Groebner Basis is a set of polynomials that spans the solution
space of another set of polynomials, they are of interest to us
because they are useful for solving systems of polynomial equations
and provide a generalized theory that shows the relationships between: 

\begin{itemize}
\item Polynomial Long Division with multiple variables and divisors,
see e.g. \cite[\S 3]{coxIdealsVarietiesAlgorithms1997}
\item The Division Algorithm see
e.g. \cite[\S 3]{coxIdealsVarietiesAlgorithms1997} and \cite{nicodemiIntroductionAbstractAlgebra2007a}
\item The LCM and GCD \cite[\S 2.6]{coxIdealsVarietiesAlgorithms1997}
\item The Euclidean Algorithm and Gaussian Elimination
\begin{itemize}
\item Both of which provide output that are special cases of
Groebner Bases.
\end{itemize}
\end{itemize}

The theory of Groebner Bases even provides a framework to
re-express the Fundamental Theorem of Algebra
\cite{prof.berndsturmfelsIntroductionGrobnerBases2017} .

\section{Polynomials}
\label{sec:org39547bd}
Let \(K\) be some field (typically \(\mathbb{Q}, \mathbb{R},
\mathbb{C}\), see \S \ref{sec:org5f09c06} for more information)

\subsection{Monomials}
\label{sec:org6536914}
A \emph{monomial} in the variables \(x_1, x_2, \ldots x_n\) is given by: \cite[p. 3]{hibiGrobnerBasesStatistics2014}

\[ 
   \begin{aligned}
   \prod^n_{i=1} \left[ x_i^{a_i} \right] = x_1^{1_1} \cdot x_2^{a_2} \cdot
   x_3^{a_3} \ldots x_n^{a_n} \quad : a \in \mathbb{Z^+}
   \end{aligned}\]

Note however that \(a\) must be a non-negative integer \cite[p. 48]{e.h.connellElementsAbstractLinear2001}
\subsubsection{Degree}
\label{sec:orga195d13}
The degree is given by the sum of the exponents, so:

$$
    \mathrm{deg}\left(   \prod^n_{i=1} \left[ x_i^{a_i} \right]   \right) =
    \sum^{n}_{i= 1}   \left[ a_i \right] 
    $$

\subsubsection{Terms}
\label{sec:orgb84041e}
A term is a monomial with a non-zero coefficient, so for example:

$$
    17 \cdot x_1^3 \cdot x_2^5 \cdot x_3^{13}
    $$

Is a term with degree 21 \((3+5+13)\) and a coefficient of 17.
\subsubsection{Polynomials}
\label{sec:org2429ee0}
A polynomial is a finite sum of terms, the degree of which
is defined to be the maximum degree of any of the terms.

\paragraph{Exception}
\label{sec:org3332be8}
The polynomial:

$$
     f = 0
     $$

Has an undefined degree. Terms only have a \textbf{non-zero} coefficient,
hence \(0\) doesn't have any terms and so the definition of degree
doesn't work for it.

Whereas \(f=c, \quad \exists c \in \mathbb{C}\) does have 1 term,
for which the degree is 0.

\paragraph{Support of a polynomial}
\label{sec:org61c2f60}
The support of a polynomial \(f\) is the set of monomials
appearing in \(f\), e.g. for the following 6th degree polynomial
in 2 variables, the support of that polynomial is given by:

\[
     f(x) = x^2+3x^3+4y \implies \mathrm{supp}\left(f\right) =
     \left\{x^2, 3x^3, 4y\right\}
     \]

The initial of the support \(\mathrm{in}_{\prec}\left(f\right)\)
is the polynomial with the highest ranking with respect to some
ordering of the monomials (see \S ) \cite[1.1.5]{hibiGrobnerBasesStatistics2014}.



\subparagraph{Other Terminology}
\label{sec:orgd96e70d}

The following terms are commonly used: \cite[\S 2.2]{coxIdealsVarietiesAlgorithms1997}

\begin{itemize}
\item The \(\mathrm{multidegree}\left(f\right)\), is the
largest power of any variable of any monomial in a polynomial
\item The Leading coefficient \(\mathrm{LC}\left(f\right)\) is the
term corresponding to the monomial containing the variable
that corresponds to the multidegree
\item The Leading monomial \(\mathrm{LM}\left(f\right)\) is the
monomial corresponding containing the variable
that corresponds to the multidegree
\item The Leading term \(\mathrm{LT}\left(f\right)\) is the product
of the leading coefficient and the leading monomial
\end{itemize}

So for example, in the polynomial:

\[
	f= 4x^2y^2 + 3x^3 + 7xy
	\]

\begin{itemize}
\item The initial is \(4x^2y^2\)
\item The Leading Coefficient is 3
\item The Leading Monomial is \(x^3\)
\item The Leading Term \footnote{This also lines up with \texttt{sympy}'s \texttt{LT()} function, beware not to
confuse the initial with the leading term, different algorithms or
ways to calculate an \(S\)-polynomial seem to use either and it
doesn't matter, I'm not sure why yet, but I am sure that there is a
difference between the initial monomial and the leading term.} is \(x^3\)
\end{itemize}




\paragraph{Homogenous Polynomial}
\label{sec:org197934c}
If all terms of a polynomial have an equal degree (say \(\exists q
     \in \mathbb{N}\) Then that polynomial is said to be a \emph{homogenous
polynomiial of degree \(q\)}, e.g.

$$
       x_1^{3}\cdot x_2^{4} \cdot x_3^{2} + x_1^{6}\cdot x_5^{2} \cdot x_7
     $$

is a homogenous polynomial of degree 7.

\paragraph{The Polynomial Ring}
\label{sec:orgde6dd2c}

The Rings, Vectors and Polynomials

Let \(K\left[x_1, x_2, x_3, \ldots x_n\right]=K\left[\mathbf{X}_n\right]\) denote the set of
all polynomials in the variables \(x_1, x_2, x_3, \ldots x_n\)
with coefficients in some field \(K\).

If \(f\) and \(g\) are polynomials from \(K\left[x_1, x_2, x_3,
     \ldots x_n\right]\) with addition and multiplication defined in
the ordinary way (i.e. just normal algebra), then \(K\left[x_1,
     x_2, x_3, \ldots x_n\right]\) forms an algebraic structure known
as a Ring.

Readers  may be familiar with the axioms of a vector space, for
which the set of polynomials in  \(K\left[x_1\right]\) of degree
less than \(n\) also satisfies \cite[\S 4.4]{larsonElementaryLinearAlgebra1991a}, a ring structure is much
the same concept, it's a set with specific characteristics.
One of the main differences is that while a vector
space requires a scalar multiplicative identity, a ring
structure does not.

On the other hand not all vector spaces are necessarily rings
because they are not necessarily closed under multiplication
(although defining multiplication by element-wise product would
remedy this), see \S \ref{sec:orgb572ffe} for more information.

\section{Ideals and Varieties}
\label{sec:org2ee65a9}
\subsection{Affine Space}
\label{sec:org32570f6}
The affine \(n\)-space of some field \(K\) is given by: \cite[\S 1.1]{coxIdealsVarietiesAlgorithms1997}

\[
   K^n=\left\{\left(a_1, a_2, a_3, \ldots, \a_n\right) \mid a_i \in K, \forall i \in \mathbb{Z}^+\right\}
   \]

For example if \(K\) was given by \(\mathbb{R}\) the resulting
affine \(n\)-space would be the \emph{Cartesian Plane}.

\subsection{Zero Point}
\label{sec:org08d0ea0}
The zero-point of some function \(f\in K\left[\mathbf{X}_n\right]\) is a point in \(K^n\):  \cite{hibiGrobnerBasesStatistics2014}

\begin{align*}
      f\left( a_1, a_2, a_3 \ldots a_n \right) =0
.\end{align*}

In the broader context of equations rather than specifically
functions, these points are often referred to as roots.

These points are often referred to as roots
\cite[\S 17.2]{judsonAbstractAlgebraTheory2016}, however this is
usually in the context of equations more broadly rather than
functions specifically. \cite{82645}
\subsection{Variety}
\label{sec:org7e52766}
Consider a set of functions \(F=\left\{ f_{1},f_{2},f_{3},\ldots
   f_{s}\right\}\), the variety of this set of functions is denoted
\(\mathbf{V}\left(F\right)\) and is the set of all zero-points of
all the functions:

\[
   \boldsymbol{V}\left(F\right)=\left\{ \left(a_{1},a_{2},a_{3},\ldots a_{n}\right)\in K^{n}\mid f_{i}\left(a_{1},a_{2},a_{3},\ldots a_{n}\right),\forall i\in\mathbb{Z}^{+}<s\right\} 
   \]

The convention is that all functions in \(F\) are set to be equal
to 0, and if this convention is taken, the variety of that set is
the set of solutions corresponding to that set of equations.
\subsubsection{Example}
\label{sec:org869ba3a}
Consider for example the set \(\left\{ -y+x^{2}-1,-y+1\right\}\),
the solution to this system can be found by substitution:


 \begin{align*}
    -y + x^{2}-1	&=0=-y+1 \\
    x^{2}-1	&=y=1 \\
    x^{2}	&=2   \\
    x	&=\pm\sqrt{2}
\end{align*}

and so:

\[
   \boldsymbol{V}\left(\left\{ -y+x^{2}-1,-y+1\right\}\right)=\left\{ \left(-\sqrt{2},1\right),\left(\sqrt{2},1\right)\right\} 
   \]

\subsection{Ideals}
\label{sec:org4222d78}
Ideals are a set with a particularly convenient property, given
functions \(f,g\in K\left[\mathbf{X}_n\right]\), a subring
\(I\subset K\left[\textbf{X}\right]\) is said to be an ideal if it
is closed under addition and admits other functions under
multiplication: \cite[\S 1.1.3]{hibiGrobnerBasesStatistics2014}

\begin{enumerate}
\item \(f\in I \land g \in I \implies f+g \in I\)
\item \(f\in I \land g \in k\left[ \textbf{X} \right] \implies gf \in I\)
\end{enumerate}

So for example, \(\left\{0\right\}\) is an ideal of the polynomial ring in
all variables, and as a matter of fact \(0\in I\) for all ideals of
polynomial rings in all variables.

A subring is a subset that is itself a ring, so \(I\) would be a
subset that is closed under addition and multiplication and
contains an additive identity (i.e. \(0 \in I\)). \footnote{It would also be sufficient to show that the \(I\) is closed
under both addition and subtraction \cite[\S 16.1]{judsonAbstractAlgebraTheory2016}} As a
matter of fact it can be shown that:

\begin{itemize}
\item \(0\in I\)
\item \(\left\{0\right\}\) is an ideal
\end{itemize}

for all ideals in all variables and that is an ideal (because
otherwise the result would not be admitted to \(I\)).
\subsubsection{Example}
\label{sec:org7443404}

Let \(R = \mathbb{Z}\) and \(I=2\mathbb{Z}\), the set of
\(\mathbb{Z}\) is a commutative ring with unity, \(2\mathbb{Z}
    \subset \mathbb{Z}\) is:

\begin{enumerate}
\item \(2\mathbb{Z} \neq \emptyset\)
\item closed under multiplication and addition
\item admits any other integer under multiplication (i.e. even
\(\times\) anything is even)
\end{enumerate}
\subsection{Ideals and Varieties}
\label{sec:orga11e9e1}
 If we have a variety of \(V \subset K^n\), we denote, \(I\left( V \right)\) as the set of all
polynomials \(f_i\in k\left[ \textbf{X} \right]\) : \cite[\S 1.1.3]{hibiGrobnerBasesStatistics2014}

\begin{align*}
      f_i\left( a_1, a_2, a_3, \ldots a_n \right) =0, \quad \forall \left( a_1, a_2, a_3, \ldots a_n \right) \in V
.\end{align*}


this set of functions satisfies the properties of an ideal and is known as the ideal of V \cite{coxIdealsVarietiesAlgorithms1997}.

In other words, the ideal of the variety of a set of functions,
\(I\left( \mathbf{V}\left(F\right) \right)\), is the set of,
polynomials, that have the same zero-points as the simultaneous zero
points of all functions in \(F\).

\subsection{Generating Ideals}
\label{sec:org509471c}
The ideal generated by \(F\) is:

\[
    \left\langle F\right\rangle =\left\{
    p_{1}f_{1}+p_{2}f_{2}+p_{3}f_{3}+\ldots p_{n}f_{n}\mid f_{i}\in
    F,p_{i}\in K\left[\boldsymbol{X}\right],\forall
    i\in\mathbb{Z}^{+}\right\}
    \]

Such a set satisfies the properties of an ideal and is a subset
of the functions that share the zero-points with \(F\): \cite[p. 34]{coxIdealsVarietiesAlgorithms1997}

\[
    \left\langle F \right\rangle \subseteq
    I\left(\boldsymbol{V}\left(F\right)\right)
    \]


\(\left\langle F \right\rangle\) is the set of all the linear combinations of elements in \(F\)
with polynomials in \(K\left[\mathbf{X}_n\right]\), another way
to phrase it would be that \(\left\langle F\right\rangle\) is the
set of polynomial consequences of \(F\) \cite[p. 30]{coxIdealsVarietiesAlgorithms1997}.

If some \textbf{finite} set of polynomials \(F\), can generate an ideal
\(I\), it is said that \(I\) is finitely generated and that \(F\)
is a basis for \(I\). Every ideal in
\(K\left[\mathbf{X}_n\right]\) is finitely generated
\cite[p. 77]{coxIdealsVarietiesAlgorithms1997}, this is known
as \emph{Hilbert's Basis Theorem}, this is important because it means we
if we had an algorithm that involved taking different polynomials
from such a basis, that algorithm would eventually end.

If two sets are bases of the same ideal, they will have the same
variety, i.e. if two sets can generate the same set of functions,
they'll have the same solutions (assuming that the set of
functions is an ideal), this also implies


\subsubsection{Initial Ideal}
\label{sec:org445b6b1}
The initial ideal:

\[\left\langle
  \mathrm{in}_{\prec}\left(I\right)\right\rangle =\left\langle
  \left\{ \mathrm{in}_{\prec}\left(f\right):0\neq f\in I\right\}
  \right\rangle \]

is generated by infinitely many monomials, namely
the initial monomials, for the infinitely many polynomials in the
ideal I. \cite[\S 1.1.5]{hibiGrobnerBasesStatistics2014}

It's common also to see a similar definition for the ideal
generated by the leading terms is denoted \(\left\langle
  \mathrm{LT}\left(f\right)\right\rangle\) \cite[\S 2.5]{coxIdealsVarietiesAlgorithms1997}.

\subsubsection{Comparison with Linear Algebra}
\label{sec:org76c6b95}
If \(S\) is some set of vectors and every vector in a vector
subspace \(V\) can be written as a linear combination of the
elements of \(S\) is is said that \(S\) spans \(V\), so for
example \(S=\left\{ \left\langle 1,0\right\rangle ,\left\langle
    0,1\right\rangle \right\}\) spans \(\mathbb{R}^2\) or
\(S=\left\{1, x, x^2\right\}\) spans \(P_2\).

Ideals for rings are similar in nature to vector subspaces and
normal subgroups. It's worth drawing attention to the fact that
that the term basis in the context of an ideal (which could be
more accurately called a generating set
\cite{sturmfelsSolvingSystemsPolynomial2002}) is quite different
from a linear basis \cite[p. 35]{coxIdealsVarietiesAlgorithms1997}.

In linear algebra a basis spans and is linearly independent, the
basis of an ideal however only spans, there is no independence,
for example:

$$\begin{aligned}
  f_{1}\left(x,y\right)=y\quad  \quad & \vec{v}_{1}=\left\langle 0,1\right\rangle \\
  f_{2}\left(x,y\right)=x \quad \quad & \vec{v}_{2}=\left\langle
  1,0\right\rangle \end{aligned}$$    

Linear independence is generally satisfied if linear
combination is equal to zero, only if the multiplying terms are
zero, i.e. \(f_1\) and \(f_2\) are linearly independent only if:


$$\begin{aligned}
  0 & =a\left\langle 0,1\right\rangle +b\left\langle 1,0\right\rangle ,\quad\forall a,b\in\mathbb{R}\\
  & =\left\langle a,b\right\rangle \\
  & \implies a=b=0\end{aligned}$$    


This clearly doesn't work for polynomials, however, because setting \(g_{i}=x\) and
\(g_{j}=-y\) satisfies such an equation.

$$0=g_{i}y+g_{j}x,\quad\not\!\!\!\implies g_i=g_j=0, \quad \forall g_{i}g_{j}\in
    k\left[\mathbf{X}\right] $$


So linear independence doesn't have a lot of meaning with polynomials, 
it's only the spanning property that is meaningful.
\section{Initials and Leading Monomials}
\label{sec:orgec30a02}
\subsection{Monomial Ordering}
\label{sec:orga3337ef}
Monomials are ordered by degree, e.g. \(x \prec x^2\) or \(xyz
   \prec x^2yz\), however in many variables it isn't always clear
which order should be chosen, for example the following monomials
have the same degree and if they are ordered by the value on first variable:

\[
   xy^3 \prec x^2yz 
   \]

If however they are ordered by trying to minimize the last
variable:

\[
   x^2yz \prec xy^3
   \]

Recall from polynomial long division that the first term in a
polynomial important to the algorithm, for a similar reason it is
necessary to decide before hand on an ordering, and generally in
this report the lexicographic order (i.e. alphabetical) will be
used.

This isn't as important as many texts make it out to be and so
further discussion appears further below in \S \ref{sec:org5a0a339}.
\section{Groebner Bases}
\label{sec:orgd4f4ac5}
A finite subset \(G\) of an ideal \(I\) is a Grobner Basis, (with
respect to some term order \(\prec\), if: \cite{berndsturmfelsIntroductionGrobnerBases2017a,hibiGrobnerBasesStatistics2014}

\[
    \left\{ \mathrm{in}_{\prec}\left(g\right)\mid g\in G\right\} 
    \]

generates \(\left\{ \mathrm{in}_{\prec}\left(I\right)\right\}\)

It's common also to see this definition reformulated with respect
to leading terms as opposed to initial monomials, in which case
\(G\) is said to be a Groebner Bases if: \cite[2.5]{coxIdealsVarietiesAlgorithms1997}

\[
    \mathrm{LT}\left(I\right)=\left\langle \mathrm{LT}\left(g_{1}\right),\mathrm{LT}\left(g_{2}\right),\mathrm{LT}\left(g_{3}\right),\ldots\mathrm{LT}\left(g_{n}\right)\right\rangle 
    \]

there are many such generating sets, we can add any element to G
to get another Groebner Basis, so in practice we may be more
concerned with reduced Groebner Basis. Note also that even though
the leading term is different from the initial monomial, either
can be used to define a Groebner Bases, however it is not yet
clear to me if the Groebner Bases will depend on the monomial
ordering \(\prec\) only if the initial is used to define it.

The variety of a set of functions depends only on the ideal of
\(F\), if two sets generate the same ideal they have the same
variety and if \(G\) is
a Grobner Basis for F, then \(V(G)=V(F)\). 

The reason we care about a Groebner Bases more generally is
because the set tends to provide more information of the solution space.

\section{Buchberger's Criterion}
\label{sec:orgb4fdca1}
\(G\) is a Groebner basis, if and only if, every \(S\)-polynomial
formed by any two pairs from \(G\) has a remainder of 0, where
the S-polynomial is given by: \cite[\S 2.6]{coxIdealsVarietiesAlgorithms1997}

\[
    S\left(f,g\right)=\mathrm{lcm}\left(\mathrm{LM}\left(f\right),\mathrm{LM}\left(g\right)\right)\times\left(\frac{f}{\mathrm{LT}\left(f\right)}-\frac{g}{\mathrm{LT}\left(g\right)}\right)
    \]
The remainder that we are concerned with is:

\[
    r = {\overline{S(f,g)}^{_G}} = S(f,g) \mod \prod_{g\in G} \left(G \right)
    \]
\section{Bucherger's Algorithm}
\label{sec:org5742e82}
Buchberger's Algorithm takes a set of polynomials, \(F\) and
eventually returns another set \(G\) which is a Groebner Bases.

To do this the algorithm tests every pair of polynomials in F with
the criterion above, if the remainder for any pair is non zero,
it is placed into \(F\) as another polynomial. 
Once every combination has been considered, the original set
\(F\) will be a Groebner Basis.

\subsection{Reduced Groebner Basis}
\label{sec:org8d408a4}
A reduced Groebner Basis is a Groebner Basis that has needless
polynomials discarded, I have not had time to investigate these
yet.

\subsection{Examples}
\label{sec:orge2124cf}
for examples of Buchberger's Algorithm, refer to the attached
\emph{Jupyter Notebook}, this is quite sparse as resources to understand
the algorithm were discovered quite late in the investigation as
was the realisation that use \texttt{sympy} had a significant amount of
documentation on the algorithm.
\section{Abstract Algebra}
\label{sec:org3828c0d}
The following are concepts that are \emph{nice to have} in understanding
the topic, but are not strictly necessary to get a broad
understanding of the topic.

They were needlessly investigated early on because accessible
resources
(e.g. \cite{coxIdealsVarietiesAlgorithms1997,andreasschulzIntegerProgrammingCombinatorial,sympydevelopmentteamSympyPolysGroebnertools})
had not yet been discovered.
\subsection{Background}
\label{sec:orga0d603b}
\subsubsection{Algebra}
\label{sec:orgc7a5871}
\paragraph{Relations}
\label{sec:org743ec47}
A relation on a set \(A\) is a subset \(R\) of the Cartesian
product:

\[
     A\times A=\left\{ \left(a,b\right):\enspace a,b\in A\right\}
     \]

If \((a,b)\in R\) it is written that \(a\enspace R \enspace b\).
\subparagraph{Example}
\label{sec:orga52d81f}
The example most relevant to the theory of Groebner bases \footnote{Relevant because we need to decide on an ordering relation in
order to use Buchberger's algorithm, which is needed to find a
Groebner Basis.} is
the \texttt{<} relation. If we had the set \(A = \left\{ 1,\ 2,\ 3 \right\}\)

The cartesian product would be:

  \begin{align*}
A\times A=\Bigg\{	&\left(1,1\right),\left(1,\ 2\right),\left(1,3\right), \\
			&\left(2,1\right),\left(2,2\right),\left(2,3\right), \\
			&\left(3,\
1\right),\left(3,2\right),\left(3,3\right)\quad\Bigg\}
\end{align*}

The set corresponding to the relation < would be:

\(\left\{ \left( 1,2 \right),\ \left( 1,3 \right),\ \left( 2,3 \right) \right\}\)

and so it is said that:

\begin{itemize}
\item \(1<2\)
\item \(1<3\)
\item \(2<3\)
\end{itemize}

\subparagraph{Types of Relations}
\label{sec:org6926688}

\begin{itemize}
\item \textbf{Reflexive} relations are relations where
\begin{itemize}
\item \(\ \forall\ a \in A, a\enspace R \enspace a\)
\end{itemize}
\item \textbf{Symmetric} relations are such that
\begin{itemize}
\item \(\forall\ a,b \in A, a\ R\ b \Rightarrow b\ R\ a\)
\end{itemize}
\item \textbf{Transitive} relations are such that
\begin{itemize}
\item \(a\ R\ b \land \ \ b\ R\ c \Rightarrow \ a\ R\ c\)
\begin{itemize}
\item \(\forall\ a,b,c \in A\)
\end{itemize}
\end{itemize}
\end{itemize}

If all of these are satisfied, the relation is said to be \emph{an
equivalence relation}.

\subparagraph{Why?}
\label{sec:org8b897fd}
Although this might seem needlessly pedantic, the algorithm we
hope to use to find solutions to systems of polynomial
equations, Buchberger's Algorithm, require us to decide on a
way to order polynomials, for example in a quadratic equation
it's pretty straight forward:

\[
      f(x) = ax^2 + bx +c
      \]

But for multiple variables it gets confusing, for example we could
order the terms by degree, but if multiple terms are of the same
degree then we could make sure that the left most variable has an
exponent that is descending, or, we could try and make sure that
the right most term is ascending:

\begin{align}
 f\left(w,x,y,z\right)	&=wz+xy \\
			 &=xy+wz
\end{align}

This is already pretty confusing so having a firm definition of
ordering is important.

\paragraph{Congruence}
\label{sec:org613b72d}
\subparagraph{Equivalence Classes}
\label{sec:orge6d7a52}
The set of all elements of \(A\) that satisfy the relation for
\(a\) is said to be the /equivalence class of \(a\) with respect to \(R\):

\[\left\lbrack a \right\rbrack_{R} = \left\lbrack a \right\rbrack = \left\{ b \in A:b\ R\ a \right\}\]

So returning to the example from \S \ref{sec:orga52d81f}, we would have:

\begin{itemize}
\item \([1]_<=\emptyset\)
\item \([2]_<=\left\{1\right\}\)
\item \([3]_<=\left\{1, 2\right\}\)
\end{itemize}
\subparagraph{Congruence Modulo \(n\)}
\label{sec:org8e95ae6}
It is said that \(a\) and \(b\) are \emph{congruent modulo \(n\)} if
\(n\mid \left(a-b\right)\) and it is written:

\[
      a\equiv b \pmod{n}
      \]
It is common to see \(\mod\) used as an operator:

\[
      a \mod b = r
      \]


The congruence class of \(a\) modulo \(n\) is expressed
\(\left[a\right]_n\) and is the equivalence class of \(a\) whereby
the relation is congruence in modulo \(n\):

\[\left\lbrack a \right\rbrack_{n} = \{ b\mathbb{\in Z\ :}b \equiv a\ \left( \text{mod\ n} \right)\}\]   


\begin{enumerate}
\item Example
\label{sec:org35225fe}
Clock time is a congruence class, for example 11 O'clock + 3 hours
= 2 PM:

\[
       \left[11\right]_{12}+\left[3\right]_{12}=\left[2\right]_{12}
       \]

Another example could be binary:


\[
       \left[1\right]_{2}+\left[3\right]_{2}=\left[0\right]_{2}
       \]


See also \cite[\S 4c]{roberthowlettUndergraduateCourseAbstract}

\item Congruence generalised with Groups
\label{sec:org618b634}
If \(G\) is a group and \(H\) a subgroup, if we have:

\[
       a^{-1}b \in H
       \]

then it is said:

\[
       a \equiv b \pmod{H}
       \]

the use of "\(\equiv\)" is appropariate because the relationship
is:

\begin{itemize}
\item reflexive
\item symmetric
\item transitive
\end{itemize}

and is hence an equiv class.

consider for example:

\[
       12 \mathbb{Z} \leqslant \mathbb{Z}
       \]

so we have 5-17 \(\in\) 12 \mathbb{Z}

So we write:

\[
       5 \equiv 17 \pmod{12\mathbb{Z}}
       \]

See \cite[\S 3.7]{gregoryleeAbstractAlgebra2018}.

\item Congruence Modulo an Ideal
\label{sec:orgb24e478}
Congruence can be extended to an ideal on any ring structure,
that's why we needed to generalise this structure, in order to use
these theoryies.

congruence modulo an ideal is

If I is an ideal in a ring R

\[
       a\equiv b\pmod{I}\iff a-b\in I
       \]

The use of \equivis justified because this is an equivalence relation

The equivalence class is the set of all elements that satisfy that
relation for \(a\):


\begin{align*}
      \forall a \in A,& \\
                      &\left[ a \right]_R = \left[ a \right] = \left\{b \in A : b r a \right\}
.\end{align*}

So in the context of congruence:

\begin{align*}
      \foral a \in G &\\
                     & \left[ a \right] = \left\{b\in G : b\equiv a \pmod{H}\right\} 
.\end{align*}

if we wanted to find \(b\) :

\begin{align*}
      b &\equiv a \pmod{H}\\
      a^{-1}b &\in H \\
      a^{-1}b &= h, \quad \exists h \in H \\
      b &= ah
.\end{align*}

So we have:

\begin{align*}
      \left[ a \right] = \left\{ah : h \in H\right\} 
.\end{align*}

This is known as the left coset
\cite[\S 6.1]{judsonAbstractAlgebraTheory2016}. The left cosets of \(H\) 
in \(G\) partition G: \cite[\S 3.3]{gregoryleeAbstractAlgebra2018}

\begin{enumerate}
\item Each \(a\in G\) is in onlyone left coset, which is \(aH\)
\item \(aH \cap bH = \emptyset\) or  \(aH=bH\)
\end{enumerate}

This can be used to show:

\begin{align*}
      H \leq G \implies \left\lvert H \right\rvert \mid \left\lvert G \right\rvert
.\end{align*}

this is known as Lagranges Theorem. \cite[\S 3.7]{gregoryleeAbstractAlgebra2018}


\begin{enumerate}
\item Normal Subgroups
\label{sec:org9dc7199}

A normal subgroup is a subgroup \(N \leq G\) :

\begin{align*}
      aN= Na \quad \forall a \in G
.\end{align*}

This is not so strict as to require all elements be commutative (although
commutative groups are of course normal)
\end{enumerate}

\item Congruence Classes for Polynomials
\label{sec:org2adce8d}
If \(f\) and \(g\) are in an ideal \(I\), then \cite[p. 240]{coxIdealsVarietiesAlgorithms1997}:

\[
       f-g \in I \implies f \equiv g \pmod{I}
       \]
\end{enumerate}
\paragraph{Groups}
\label{sec:org5e14bec}
A set \(G\) is a group, if there in a binary operation, \(\star\),
defined on that set such that:

\begin{enumerate}
\item The binary operation is closed on the set
\[a,b \in G \implies a\star b \in G\]
\item The binary operation is associative

\[a,b,c \in G \implies a\star (b\star c) = (a\star b)\star c\]
\item There is an element that doesn't do anything under the binary
operation, this is known as an identity element, for example 1 is
an identity element to the multiplication operation.

\begin{align*}
\exists e \in G:&\\
		& a\star e = e \star a = a
\end{align*}
\item Every element has an inverse

  \begin{align*}
  \forall a \in G,\enspace \exists a^{-1} \in G:	&\\
						   & a\star a^{-1} = e
\end{align*}

\begin{itemize}
\item For operations that are additive in nature, it is common to use
the notation: \(-a\) \cite[\S 3.3]{gregoryleeAbstractAlgebra2018}
\end{itemize}
\item If the binary operation is also commutative, the group is said to be abelian:

\begin{align*}
\forall a,b \in G,& \nonumber \\
		& a \star b = b \star a \iff G \text{ is abelian.} 
\end{align*}
\end{enumerate}

\subparagraph{Example}
\label{sec:orgb09c6ab}

An example of a group is a set of all matrices of a given size under addition,
this can be seen because:

\begin{enumerate}
\item Adding matrices gives back matrices of the same size,
\item Introducing brackets in addition doesn't change the result
\item A matrix with all 0's is an identity
\item Any matrix \(\mathbf{A}\) has an inverse (namely \(-\mathbf{A}\))
\end{enumerate}

This example would also be an abelian group because addition is commutative.

Note that if the operation was matrix multiplication, \(\cdot\)
(denoted as \texttt{\%*\%} in \textbf{\emph{R}}
\cite{rcoreteamLanguageEnvironmentStatistical2020}), only square
matrices with a non-zero determinant
(e.g. \(\left\lvert\mathbf{A}\right\rvert \neq 0\)) could be a
group. This is because the matrix would need to be invertible. \footnote{although the element-wise product, \(\odot\), would not present this issue.}

\subparagraph{But Why?}
\label{sec:orgcb36d8d}
The reason groups are interesting is because many natural
structures can be described by a set and a binary operation,
obvious examples are sets of numbers, vectors, matrices and
equations, but more generally Group theory can be used to describe
puzzles like \emph{Rubik's Cube} \cite{joynerAdventuresGroupTheory2002},
chemical structures \cite{GroupTheoryIts2013} and has been used in
the theory of
quantum mechanics \cite{tinkhamGroupTheoryQuantum2003}. \footnote{See generally this \cite{19328} \emph{Stack Exchange Discussion}.}

\paragraph{Rings}
\label{sec:orgb572ffe}
Examples, equivalence class ring
\cite[Ch. 3]{judsonAbstractAlgebraTheory2016} see also \S 2.4 of
nicodemii \cite[\S 2.4]{nicodemiIntroductionAbstractAlgebra2007a}

Rings are an abelian group under addition \(+\), with a second binary
operation that corresponds to multiplication \(\times\), this
operatuion must be closed, associative and distributive, but there is
no need for an inverse or identity
\cite[\S 8.1]{gregoryleeAbstractAlgebra2018}. So a ring structure
is a set \(\mathcal{R}\), with two closed binary operations, that
satisfies the following axioms of a ring
\cite[\S\S 2.4-2.6]{nicodemiIntroductionAbstractAlgebra2007a}:

\begin{enumerate}
\item Associativity of Addition

\(\left( \forall a,b,c \in \mathcal{R} \right) \left( a+ b \right) +  c = a +  \left(  b +  c    \right)\)
\end{enumerate}


\begin{enumerate}
\item Commutativity of Addition

\(\left( \forall a,b \in \mathcal{R}  \right) a +  b = b +  a\)

\item Additive Elements Exist

\(\left( \forall a \in \mathcal{R} \right) \wedge \left( \exists 0 \in \mathcal{R} \right) a +  0= 0 +  a =  a\)

\item Additive Inverse Exists

\(\left( \forall a \in \mathcal{R} \right)\wedge \left( \exists b \in \mathcal{R} \right) a +  b =  b +  a = 0\)

\begin{itemize}
\item This can be equivalently expressed:
\end{itemize}

\(\left( \forall a \in \mathcal{R} \right)\wedge \left( \exists \left( - a\right)\in \mathcal{R} \right) a +  \left( - a \right) = \left( - a \right) +  a = 0\)

\item Associativity of Multiplication

\(\left( \forall a,b,c, \in \mathcal{R} \right)\left( a \cdot  b \right)\cdot c = a \cdot  \left( b \cdot  c \right)\)

\item Distributivity of Multiplication over Addition

\begin{itemize}
\item \(\left( \forall a,b,c, \in \mathcal{R} \right) \left( a\cdot  \left( b+ c \right)=  \left( a \cdot   b  \right) +  \left( a \cdot   c  \right) \right)\), AND
\item \(\left( \forall a,b,c, \in \mathcal{R} \right)\left( a +  b \right)\cdot   c = \left( a \cdot   c  \right)+  \left( b \cdot   c \right)\)
\end{itemize}
\end{enumerate}
\subparagraph{Further Axioms}
\label{sec:orgea4f752}

Other conditions that correspond to different classes of rings are:

\begin{enumerate}
\item Commutativity of Multiplication
\begin{itemize}
\item A ring that satisfies this property is called a \textbf{commutative ring}

\(\left( \forall a,b \in \mathcal{R} \right) a \cdot  b = b \cdot  a\)
\end{itemize}

\item Existence of a Multiplicative Identity Element (A ring with Unity)
\begin{itemize}
\item A ring that satisfies this property is called a \textbf{ring with identity} or
\end{itemize}
equivalently a \textbf{ring with unity} (the multiplicative identity, often
denoted by \(1\), is called the \textbf{unity} of the ring.

\(\left( \exists 1 \in \mathcal{R} \right) \left( \forall a \in \mathcal{R} \right) 1 \cdot  a = a \cdot  1 = a\)
\end{enumerate}
\subparagraph{Example}
\label{sec:orge6fc582}
An obvious example of a ring is the set of all integers
\(\mathbb{Z}\) with the ordinary meaning of addition and
multiplication. A more insightful example would be a congruence
class, for example \(\mathbb{Z}_{12}\), this satisfies the axioms
of a ring, but some values are zero divisors. If two elements of a
ring multiply to give 0, those values are said to be zero divisors,
for example 3 and 4 are zero divisors in \mathbb{Z}\textsubscript{12}:

\[
       \left[3\right]_{12}\times\left[4\right]_{12}=\left[0\right]_{12}
      \]

An element that has an inverse is said to be a unit, for example:

 \[
     \left[2\right]_{9}\times\left[5\right]_{9}=\left[1\right]_{9}
     \]
An element can't both be a unit and a zero divisor, because one
multiplies to give 0 and the other to give 1, however, in many
algebraic structures (e.g. \(\mathbb{Q}, \mathbb{R}\) or
\(\mathbb{C}\)) every element has a multiplicative inverse, and
this motivates the next algebraic structure. 

\paragraph{Integral Domains}
\label{sec:org75b9387}
An integral domain is a commutative ring with identity that has no
zero divisors.
\subparagraph{Example}
\label{sec:orgce15308}
The obvious example of an integral domain is \(\mathbb{Z}\), but any
\(\mathbb{Z}_p\) where \(p\) is a prime number, will also be an integral domain.

Another example is the set of all polynomials with real
coefficients, this will be explored in greater detail below, but
for the moment observe that this algebraic structure conforms to
the axioms of a ring and has no zero divisors.

It can be clearly seen that the set of polynomials has no zero
divisors because:

\begin{align}
f \times g &= 0 \\
&\implies f = 0, \lor g = 0 \ \\
\end{align}

in either case \(f\) or \(g\) is not a non-zero divisor.

Note however that not every element of the polynomials has an
inverse, for example the function \(f(x)=x\) would have an inverse
\(f^{-1}(x)=\frac{1}{x}\), but this is not a polynomial.

This leads to the final algebraic structure that will be considered
here. \footnote{There are other algebraic structures that could be interesting,
for example polynomials can also be considered as vectors, see
e.g. \cite{larsonElementaryLinearAlgebra1991a}, as a matter of fact all
vector spaces are rings if multiplication is defined element-wise by
the \emph{Hadamard product} (\(\odot\)), this could be an interesting
relationship to investigate further.}
\paragraph{Fields}
\label{sec:org5f09c06}
A field is a commutative ring with identity in which all non-zero
elements are units.

Because every element of a field is a unit, it
implies that every element is not a zero-divisor, and so hence a
field is:

\begin{itemize}
\item a special case of an integral domain, which is in turn
\item a special case of a ring, which is in turn
\item a special case of a group.
\end{itemize}

\paragraph{Rings and Integral Domains}
\label{sec:org44ae102}
It seems that the reason the theory of Groebner Bases is concerned
with the ring of polynomials over a field is related to the
irreducibility of the polynomial, see generally \cite{EquivalenceDefinitionsIrreducible}.

Note also that the Ring of polynomials over an integral domain (a
property satisfied also by a field) is more accurately an
integral domain
\cite{sympydevelopmentteamBasicFunctionalityModule,RingPolynomialForms},
not merely a ring.
\subparagraph{Why aren't Polynomials fields}
\label{sec:orgb4f303f}
A field is an integral domain, for which every element has an
inverse, so consider some function, say \(g(x)=x\), if the set of polynomials
was a field, there would have to exist some \(f(x)\) such that:

\[
      x \cdot f(x) = 1
      \]

however if we evaluate this at \(x=0\)

\[
      0 \cdot f(0) = 1
      \]

well\ldots{} this clearly doesn't work, so it's clear that this \(f(x)\)
doesn't exist and so the set of polynomials is not a field, see
generally \cite{billdubuqueAbstractAlgebraWhy}

One might wonder if there's a good reason why \(f(x)=\frac{1}{x}\)
isn't considered a polynomial, notwithstanding the fact that it
doesn't quite fix this example with 0:

\begin{itemize}
\item All polynomials over the real numbers are continuous, that
would make this membership inconvenient.
\begin{itemize}
\item On the other hand there are discontinuities of arbitrary
polynomials over certain fields, what's a good example of
a such a field though?
\end{itemize}
\end{itemize}


The easy and uninformative answer is that \(\frac{1}{x}\) does
not have positive indices, outright violating the definition.

\subsubsection{Vector Spaces}
\label{sec:orgb05a40f}
The ring of polynomials over a field \(K\):

\[
    K\left[x_1, x_2, x_3, \ldots, x_n\right]
    \]

is a \(n\)-vector space with a basis given by the set of all power products:

\[
    \left\{x_1^{\beta_1}, x_2^{\beta_2}, x_3^{\beta_3}, \ldots x_n^{\beta_n} \right\}
    \]
\paragraph{Basis}
\label{sec:org0905961}
A basis is a set of vectors that
\cite[p. 39]{axlerLinearAlgebraDone2014} are:

\begin{itemize}
\item Linearly independent
\item Spans an \(n\)-dimensional vector space??
\end{itemize}
\subparagraph{Linear Independence}
\label{sec:org6448a14}
a set of vectors are linearly independent if:

\[
      a_{1}v_{1}+a_{2}v_{2}+a_{3}v_{3}\ldots=0 \iff a_{1}=a_{2}=a_{3}=\ldots=a_{m}
      \]
\subparagraph{Span}
\label{sec:orgbadae51}
The span of a set of vectors, is the set of all possible linear
combinations of those vectors.

So for example:
\begin{align}
\mathbb{R}^2&=\mathrm{span}\left( \left\{\left(0,1\right), \left(1, 0\right)\right\}  \right)\\
	    &=\mathrm{span}\left( \left\{\left(0,2\right), \left(2, 0\right)\right\}  \right)\\
	    &=\mathrm{span}\left( \left\{\left(1,1\right), \left(1, -1\right)\right\}  \right)\\
\end{align}

To visualize this in \(\mathbb{R}^2\), imagine that by varying
the scaling value of each vector, any point on \(\mathbb{R}^2\)
can be reached.

\paragraph{Vectors}
\label{sec:org1f848e4}
A ring with unity is a vector space, however a vector space only
needs to be closed under scalar multiplication. This means
vector spaces are not necessarily rings unless the
multiplication operation is closed, an example of a closed
vector multiplication is element-wise multiplication, this is
known as the hadamard product (think like mutliplying `numpy` arrays.)
\subsection{Monomial Orders}
\label{sec:org5a0a339}
    \href{20210406222024-groebner\_bases\_of\_a\_system\_of\_equations.org}{groebner bases of a system of equations}
A partial order on a set is a relation \(R\):

\begin{itemize}
\item \(x R x\)

\begin{itemize}
\item reflexivity
\end{itemize}

\item \(x R y \land y R x \implies x = y\)

\begin{itemize}
\item Antisymmetry
\end{itemize}

\item \(x R y \land y R z \implies x R z\)

\begin{itemize}
\item Transitivity
\end{itemize}
\end{itemize}

So for example, the set of integers has \(\leq\) as a relation such that
\(n_1\in \mathbb{Z}:\)

\begin{itemize}
\item \(n\leq n\)

\item \(n_1\leq n_2 \land n_2 \leq n_1 \implies n_1=n_2\)

\item \(n_1\leq n_2 \land n_2 \leq n_3 \implies n_1\leq n_3\)
\end{itemize}

A partially ordered set is one with a relation that is a partial order.

\begin{itemize}
\item partial order

\begin{itemize}
\item a relation
\end{itemize}

\item partially ordered set

\begin{itemize}
\item a set
\end{itemize}
\end{itemize}

A total order is a partial order such that \(\forall x,y\) either \(x R y\)
or \(y R x\), the obvious example is \(<\), consider for example
\(\mathbb{C}\), this has a partial order if the the modulus is considered,
it's only a partial order because, e.g.
\(\left\lvert i+i \right\rvert= \left\lvert i-i \right\rvert\). not all
sets will have a partial ordering, e.g. the somewhat contrived example
has no (at least obvious) partial order.

$$\left\{\square, \triangle, \sqrt{-1} x^{e^x} \right\} 
.$$

\(k\left[ \mathbf{X} \right]\) is a polynomial ring in \(n\) variables and
\(\mathcal{M}_n\) is the set of amonomials in the variables
\(x_1, x_2, x_3, \ldots x_n\).

A monomial order on \(k\left[ \mathbf{X} \right]\) is a total order on
\(\prec\) on \(\mathcal{M}_n\):

\begin{enumerate}
\item \(i \prec u, \quad \forall 1\in u\in \mathcal{M}_n\)

\item \(u, v\in \mathcal{M}_n \land u \prec v \implies uw \prec vw, \forall w \in \mathcal{M}_n\)
\end{enumerate}

\paragraph{Lexical monomial order}
\label{sec:orgca3f61a}
Let:

$$\begin{aligned}
     u &= x_1^{a_1} x_2^{a_2} x_3^{a_3} \ldots x_n^{a_n} \\
     v &= x_1^{b_1} x_2^{b_2} x_3^{b_3} \ldots x_n^{b_n}
 .\end{aligned}$$ The lexicographic order on \(k\left( \mathbf{X} \right)\)
is given by the total order \(<_{\mathrm{lex}}\) on \(\mathcal{M}_n\) by
setting:

$$\begin{aligned}
     u <_{\mathrm{lex}} v
 .\end{aligned}$$

if:

\begin{enumerate}
\item \(\sum^{n}_{i=1}\left[ a_i  \right] \leq \sum^{n}_{i=1}\left[ b_i  \right]\)

\item the leftmost non-zero term in the following vector is positive:

\begin{itemize}
\item b\textsubscript{1}-a\textsubscript{1}, b\textsubscript{2}-a\textsubscript{2}, b\textsubscript{3}-a\textsubscript{3} \ldots{}b\textsubscript{n}-a\textsubscript{n}
\end{itemize}
\end{enumerate}

Reverse lexicographic is:

\begin{enumerate}
\item \(\sum^{n}_{i=1}\left[ a_i  \right] \leq \sum^{n}_{i=1}\left[ b_i  \right]\)

\item the \textbf{rightmost} non-zero term in the following vector is \textbf{negative}:

\begin{itemize}
\item b\textsubscript{1}-a\textsubscript{1}, b\textsubscript{2}-a\textsubscript{2}, b\textsubscript{3}-a\textsubscript{3} \ldots{}b\textsubscript{n}-a\textsubscript{n}
\end{itemize}
\end{enumerate}

These should be combined into one statement \(\uparrow\)

So for example consider: $$x_1x_4-x_2x_3
 .$$

by lexicographic we have

$$x_2x_3\prec x_1x_4
 .$$

because the leftmost entry is positive in the vector described before:

$$\left\langle 1, -1, -1, 1\right \rangle
 .$$

by reverse lexicographic we have

$$x_1x_4 \prec x_2x_3
 .$$

because the \textbf{rightmost} entry is \textbf{negative} in the vector described
before:

$$\left\langle -1, 1, 1, -1\right \rangle
 .$$

This may be discussed more in the org mode note.

an interesting property that comes back in the buchberger algorithm and
polynomial long division is:

$$\mathrm{in}_{\prec}\left( f \cdot g \right) = \mathrm{in}_{\prec}\left( f \right) \mathrm{in}_{\prec}\left( g \right) 
 .$$

\paragraph{Colloquial}
\label{sec:org93e3f5b}
\subparagraph{Lexicographic}
\label{sec:org3730d87}
The highest variable is so expensive that it makes the entire
monomial expensive.
\subparagraph{Reverse Lexicographic}
\label{sec:org6c77df3}
The lowest variable is so chap that it makes the entire monomial cheap.
\subsection{Dickson's Lemma}
\label{sec:org71762fc}
\subsubsection{Divisors}
\label{sec:org2b3dd0c}
For \emph{monomials}:

\begin{itemize}
\item \(u= \prod^n_{i=1}\left[ x_i^a_i \right] \quad a \in \mathbb{Z^+}\)
\item \(v= \prod^n_{i=1}\left[ x_i^b_i \right] \quad b \in \mathbb{Z^+}\)
\end{itemize}

\(u\) is said to divide \(v\) if \(a_i \leq b_i \quad \forall i \in \left[ 1, n \right]\)

\paragraph{Example}
\label{sec:org85e8673}

Consider:

\begin{itemize}
\item \(u = x^2y^3z^5\)
\item \(v = x^1y^2z^3\)
\end{itemize}

In this case \(v \mid u\) because:

\begin{align*}
      1 &< 2 \\
      2 &< 3 \\
      3 &< 5 \\
      \ \\
      \frac{u}{v} &= \frac{x^2}{x^1} \cdot \frac{y^3}{y^2} \cdot  \frac{z^5}{z^3}
.\end{align*}

\subsubsection{Minimal Element}
\label{sec:org5f6d987}

let \(\mathcal{M}_n\) be the set of a all monomials in the variables
\(x_1, x_2, x_3, \ldots x_n\) and \(M \subset \mathcal{M}_n\) be a
nonempty subset thereof.

The following condition describes a minimal element \(u\in M\):

\[
    \left(v \in M \land v \mid u\right) \implies v = u
    \]

In other words, \(u\) is a minimal element if the only way that \(v
    \mid u\) is if \(v = u\).


\paragraph{Example}
\label{sec:orgef073aa}

Consider \(\mathcal{M}_2\):

\begin{alignat}{3}
  \mathcal{M}_2 &= \{&x  y, &x  y^2, &x  y^3, \ldots         \\
                &    &      &x^2y,   &x^2y^2, x^2y^3, \ldots \\
                &    &      &x^3y,   &x^3y^2, x^3y^3, \ldots \\
                &    &      & \vdots &                       \\
		  \}
\end{alignat}

and let's have the subset \(M = \left\{ x^2y, x^2y^2, x^2y^3 \ldots
     \right\}\), the minimum elements are:

\[
           \left\{x^2y\right\}
     \]


clearly \(\left\lvert M \right\rvert = \infty\), however this number of
mi \centernot\implies um elements will always be finite, this is known as
\textbf{Dickson's Lemma}.

\subsubsection{Dickson's Lemma}
\label{sec:orga0a8af0}

\begin{quote}
/Dickson's Lemma is the main result needed to prove the termination
of Buchberger's algorithm for computing Groebner basis of polynomial
ideals/ \cite{martin-mateosFormalProofDickson2003}.
\end{quote}

Let
\begin{itemize}
\item \(\mathcal{M}_n\) be the set of all monomials in variables \(x_1, x_2, x_3 \ldots x_n\).
\item \(M\) be a nonempty subset of \(\mathcal{M}_n\)
\end{itemize}

\begin{quote}
\emph{The set of minimal elements of a nonempty subset \(M \subset
    \mathcal{M}_n\) is at most finite.}
\end{quote}

This intuitively makes sense, I can't have an infinite number of
minimums, otherwise they wouldn't be minimums, the proof is very
difficult though.

\paragraph{In one Variable}
\label{sec:org77f5313}

By definition, a monomial is raised to the power of a non-zero
integer, in a single variable monomial the smallest index will
correspond to the minimal element (by the definition of the minimal
element) and hence the existence of a minimum element in
\mathbb{Z^+} implies the existence of a minimum element in
\(M\subset \mathcal{M}_n\).


\paragraph{In Two Variables}
\label{sec:orgc4d2085}

Assume that there is an infinite number of minimal elements:

\begin{align}
      u_1 &= x^{a_1}y^{b_1} \\
      u_2 &= x^{a_2}y^{b_2} \\
      u_3 &= x^{a_3}y^{b_3} \\
      u_4 &= x^{a_4}y^{b_4} \\
      u_5 &= x^{a_5}y^{b_5} \\
      \ldots \nonumber
\end{align}
Let's order the values by the first exponential such that \(a_1 \leq a_2 \leq a_3 \ldots\).

If \(a_i=a_{i+1}\), then either:

\begin{itemize}
\item \(u_1 = u_{i+1}\)
\begin{itemize}
\item We can't have this because set's do not contain repeated elements.
\end{itemize}
\item \(y^{b_i} \neq y^{b_{i+1}}\)
\begin{itemize}
\item But this would mean that either \(u_i\) or \(u_{i+1\}\) is
not a minimal element, so this can't occur either.
\end{itemize}
\end{itemize}


This means that each \(a_i\) must be different and so:

\begin{align}
a_i < a_2 < a_3 \ldots
\end{align}

If \(u_i | u_{i+1}\) one of them is not a minimal element and so we
must have \(b_i > b_{i+1}\), hence \(b_i > b_2 > b_3 \ldots\).

This means that \(b_i\) represents an upper bound for the number of
different minimal elements, hence the number of minimal elements
must be finite.



\paragraph{In \(n\) variables\hfill{}\textsc{induction}}
\label{sec:orgb0a1960}
If the number of minimal elements is finite for \(M_n \subset
     \mathcal{M}_n\) we would expect \(M_{n+1}\) to be finite as well,
adding an extra variable should not make the number of minimal
elements infinite because the integers in the index will still
behave as an upper bound.

I need to formalise this as per \cite[\S 1.1.2]{hibiGrobnerBasesStatistics2014}.
\end{document}